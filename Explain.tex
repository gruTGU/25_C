% !TEX program = xelatex
\documentclass[11pt,a4paper]{article}

% ---------- 基础版式 ----------
\usepackage[margin=1in]{geometry}
\usepackage{setspace}
\setstretch{1.15}
\usepackage{parskip}
\setlength{\parskip}{0.6em}
\setlength{\parindent}{0pt}

% ---------- XeLaTeX 字体与中文支持(macOS 适配) ----------
\usepackage{fontspec}
\usepackage{xeCJK}
\usepackage{ifthen}
\usepackage{xcolor}

% 英文字体(优先 Apple 系,缺省回退到 Times New Roman)
\IfFontExistsTF{Times New Roman}{
  \setmainfont{Times New Roman}
}{
  \setmainfont{Times}
}
\IfFontExistsTF{Helvetica Neue}{
  \setsansfont{Helvetica Neue}
}{
  \setsansfont{Arial}
}
\IfFontExistsTF{Menlo}{
  \setmonofont{Menlo}
}{
  \setmonofont{Courier New}
}

% 中文字体(按 macOS 常见中文字体依次回退)
\IfFontExistsTF{sans-serif}{
  \setCJKmainfont{sans-serif}
}{
  \IfFontExistsTF{STHeiti}{
    \setCJKmainfont{STHeiti}
  }{
    \IfFontExistsTF{WenQuanYi Zen Hei}{
      \setCJKmainfont{WenQuanYi Zen Hei}
    }{
      \setCJKmainfont{Arial Unicode MS}
    }
  }
}
\IfFontExistsTF{Heiti SC}{
  \setCJKsansfont{Heiti SC}
}{
  \setCJKsansfont{PingFang SC}
}
\IfFontExistsTF{Kaiti SC}{
  \setCJKmonofont{Kaiti SC}
}{
  \setCJKmonofont{Songti SC}
}

% 中文断行 & 数学内中文
\xeCJKsetup{CJKglue=\hskip 0.15em plus 0.05em minus 0.05em, CJKmath=true}

% ---------- 数学/排版宏包 ----------
\usepackage{amsmath, amssymb, amsthm, bm, mathtools}
\usepackage{booktabs}
\usepackage{graphicx}
\usepackage{siunitx}
\sisetup{detect-all = true}
\usepackage{hyperref}
\hypersetup{
  colorlinks=true,
  linkcolor=blue,
  urlcolor=blue,
  citecolor=blue
}

\title{NIPT 最优检测时点研究,Q3--Q4 方法说明}
\author{}
\date{}

\begin{document}
\maketitle
\tableofcontents

\section{符号与数据处理约定}
\subsection{符号约定}
\begin{itemize}
  \item 观测响应:$Y\in(0,1)$ 表示按比例尺度(而非百分比)的胎儿成分 proxy(男胎用 Y 染色体浓度)。
  \item 阈值:$y^\star=0.04$(即 4\%)为“有效检测”阈值;命中指标
  \[
    \text{hit} = \mathbf{1}\{Y \ge y^\star\}.
  \]
  \item 协变量:$x=(t, b, a, h, w, z)$,其中
  \begin{align*}
    t&:\ \text{检测孕周(周)},\quad
    b:\ \text{BMI},\quad
    a:\ \text{年龄},\\
    h&:\ \text{身高},\quad
    w:\ \text{体重},\quad
    z:\ \text{与测序/样本质量相关的 QC 指标(如 GC、过滤比例等)}.
  \end{align*}
  \item 分组:按 BMI 将总体分为若干区间 $g\in\mathcal{G}$,每组使用组内中位数作为“代表情景”,记为
  \(
  s_g=\bigl(b_g^{\text{med}}, a^{\text{med}}, h^{\text{med}}, w^{\text{med}}, z^{\text{med}}\bigr).
  \)
\end{itemize}

\subsection{数据清洗要点(与 Q1 一致)}
\begin{enumerate}
  \item 孕周解析:将诸如“$11\text{w}+6$”等格式统一换算为连续周 $t\in\mathbb{R}_+$。
  \item 单位统一:若大部分 $Y$ 值 $>1$,视为百分比数据并除以 $100$ 转为比例。
  \item 合理范围与质控:筛除极端/缺失;可选保留 GC 区间(如 $[0.3,0.7]$);重复检测保留首检。
  \item 男胎识别:以 $Y>0.005$ 为保守阈值判定男胎,进入后续分析。
\end{enumerate}

\section{Q3:两层模型构建 $P_{\text{hit}}(t\mid x)$}
Q3 的目标是给出随孕周变化的“达标概率”曲线
\[
  P_{\text{hit}}(t\mid x)=\mathbb{P}\bigl(Y_{\text{obs}}(t,x)\ge y^\star\bigr),
\]
并在 BMI 分组层面,给出“最早达标时点”与“风险--命中权衡时点”,同时量化不确定性与测量误差的影响。

\subsection{层 A(均值模型):$\mu(x)=\mathbb{E}[Y\mid x]$}
在原始比例尺度上拟合
\[
  \mu(x)=\mathbb{E}[Y\mid x]\approx X(x)\,\bm{\beta},
\]
其中 $X(x)$ 为以孕周 $t$ 的多项式基函数(或样条)与 BMI 及其交互构成的设计矩阵。一个常用的可解释设定为“$t$ 的 $d$ 次多项式 + 线性 BMI + 交互项 + 线性协变量”:
\[
  X(x)=\bigl[1,\, t,\, t^2,\,\dots,\,t^d,\, b,\, a,\, h,\, w,\, t\!\cdot\!b,\, t^2\!\cdot\!b,\,\dots,\,t^d\!\cdot\!b \bigr],
\]
系数 $\bm{\beta}$ 由最小二乘估计:
\[
  \widehat{\bm{\beta}}=\arg\min_{\bm{\beta}}\sum_i\bigl\{Y_i-X(x_i)\bm{\beta}\bigr\}^2.
\]
预测时有
\[
  \widehat{\mu}(x)=X(x)\,\widehat{\bm{\beta}},\qquad
  \widehat{\mu}(x)\in[\epsilon,\,0.5-\epsilon]\ \text{(数值裁剪,防溢出)}.
\]

\subsection{层 B(观测误差):$Y_{\text{obs}}=\mu(x)+\varepsilon$}
考虑测量/技术误差,设
\[
  Y_{\text{obs}}(x)=\mu(x)+\varepsilon,\qquad
  \varepsilon\sim\mathcal{N}\bigl(0,\,\sigma^2(x)\bigr),
\]
其中 $\sigma(x)$ 的估计有三种稳健方案:
\begin{enumerate}
  \item \textbf{Global:}用整体残差标准差 $\hat\sigma=\operatorname{sd}\{Y-\widehat{\mu}(x)\}$。
  \item \textbf{GA-Local:}对孕周 $t$ 分箱(如步长 $0.5$ 周),在每个箱内计算残差标准差并对 $t$ 做近邻插值,记为 $\widehat{\sigma}_{\text{loc}}(t)$;对小样本/单点箱进行全局回退与边界裁剪以保证数值稳定。
  \item \textbf{By-QC:}若有 QC 指标 $z$,以线性形式
  \(
    \sigma(x)\approx \alpha_0+\bm{\alpha}^\top z
  \)
  拟合残差标准差与 QC 的关系(特征标准化后最小二乘)。
\end{enumerate}

\subsection{达标概率与两类“最佳时点”}
由于 $Y_{\text{obs}}\sim\mathcal{N}\bigl(\mu(x),\sigma^2(x)\bigr)$,有
\[
  P_{\text{hit}}(x)=\mathbb{P}(Y_{\text{obs}}\ge y^\star)
  =1-\Phi\!\left(\frac{y^\star-\mu(x)}{\sigma(x)}\right),
\]
其中 $\Phi(\cdot)$ 为标准正态分布函数。对 BMI 组 $g$,在“组中位情景”
\(
  s_g=(b^{\text{med}}_g,a^{\text{med}},h^{\text{med}},w^{\text{med}},z^{\text{med}})
\)
下定义随孕周的曲线
\[
  P_{\text{hit}}(t\mid s_g)=1-\Phi\!\left(\frac{y^\star-\mu(t,s_g)}{\sigma(t,s_g)}\right).
\]

\paragraph{最早达标时点(阈值法)}
给定目标达标概率阈值 $\tau$(如 $0.90$),定义
\[
  t^{\text{target}}_g=\inf\bigl\{\,t:\ P_{\text{hit}}(t\mid s_g)\ge \tau\,\bigr\}.
\]

\paragraph{风险--命中权衡时点}
题面风险分档定义为
\[
\text{risk\_level}(t)=
\begin{cases}
1, & t\le 12,\\
2, & 13\le t \le 27,\\
3, & t\ge 28.\\
\end{cases}
\]
给定权衡参数 $\lambda>0$,定义组别 $g$ 的目标函数
\[
  J_g(t;\lambda)=\text{risk\_level}(t)+\lambda\bigl(1-P_{\text{hit}}(t\mid s_g)\bigr),
\]
从而
\[
  t^{\text{risk}}_g=\arg\min_t J_g(t;\lambda).
\]

\subsection{不确定性评估与误差敏感性}
\paragraph{Bootstrap 置信区间}
对每个 BMI 组 $g$,重复 $B$ 次组内重采样,\emph{每次}重拟合 $\mu,\sigma$ 并重算 $t^{\text{target}}_g$ 与 $t^{\text{risk}}_g$:
\[
  \bigl\{t^{\text{target},(b)}_g,\ t^{\text{risk},(b)}_g\bigr\}_{b=1}^B.
\]
以百分位数法给出 95\% CI(或报告均值 $\pm$ 中位绝对偏差)。

\paragraph{$\sigma$ 敏感性}
对 $\sigma$ 施加比例扰动 $s\in\{0.8,1.0,1.2\}$,重算
\(P_{\text{hit}}\)、$t^{\text{target}}_g$、$t^{\text{risk}}_g$,
以量化测量误差对建议时点的影响。

\section{Q4:检测与复检策略优化}
Q4 在 Q3 的 $P_{\text{hit}}(t\mid x)$ 基础上,联合考虑“检测成本/复检成本/拖延惩罚/未达标惩罚/孕周风险”,求解\emph{首检时点}与(可选)\emph{复检时点}的最优策略。

\subsection{单次检测(基线)}
仅允许一次抽血,候选区间 $t\in[t_{\min},t_{\max}]$。给定权重
\(
c_1\ (\text{单次成本}),\ \lambda\ (\text{未达标惩罚}),\ \kappa\ (\text{拖延惩罚/周})
\)
,定义
\[
  J_1(t)=\underbrace{\text{risk\_level}(t)}_{\text{孕周风险}}
       +\lambda\bigl(1-P_{\text{hit}}(t\mid x)\bigr)
       +c_1+\kappa\,(t-t_{\min}),
\]
取
\(
  t_1^\star=\arg\min_{t\in[t_{\min},t_{\max}]} J_1(t).
\)

\subsection{两次检测(允许复检)}
策略为 $(t_1,\Delta)$,其中二检时点 $t_2=t_1+\Delta$。记
\[
  P_1=P_{\text{hit}}(t_1\mid x),\qquad
  P_2=P_{\text{hit}}(t_2\mid x).
\]
考虑两次命中概率的相关性,令 $\alpha\in[0,1]$,并定义
\[
  P_2'=\alpha P_2+(1-\alpha)P_1,
\]
则整体成功概率
\[
  P_{\text{succ}}=P_1+(1-P_1)P_2'.
\]
期望结果孕周(若二检仍未达标,也以 $t_2$ 结束流程)为
\[
  \mathbb{E}[T_{\text{res}}]=P_1\,t_1+(1-P_1)\,t_2,
\]
期望孕周风险
\[
  \mathbb{E}[\text{risk}]=P_1\,\text{risk\_level}(t_1)+(1-P_1)\,\text{risk\_level}(t_2),
\]
期望抽血次数 $\mathbb{E}[\text{draws}]=1+(1-P_1)$。
给定 $c_1$(首检成本)、$c_r$(复检成本)、$\lambda$(未达标惩罚)、$\kappa$(延迟惩罚),
两步策略的目标函数为
\[
  J_2(t_1,\Delta)=
  c_1 + (1-P_1)c_r
  + \mathbb{E}[\text{risk}]
  + \lambda\,(1-P_{\text{succ}})
  + \kappa\,\bigl(\mathbb{E}[T_{\text{res}}]-t_{\min}\bigr),
\]
在网格
\(
t_1\in[t_{\min},t_{\max}],\ \Delta\in\mathcal{D}
\)
上搜索最小值:
\[
  (t_1^\star,\Delta^\star)=\arg\min_{t_1,\Delta} J_2(t_1,\Delta),\qquad
  t_2^\star=t_1^\star+\Delta^\star.
\]

\subsection{输出与解释}
对每个 BMI 组 $g$(或个体 $x$),报告
\begin{itemize}
  \item 单次策略:$t_1^\star$、成功率 $P_{\text{hit}}(t_1^\star\mid x)$、目标值 $J_1(t_1^\star)$;
  \item 两次策略:$(t_1^\star,\Delta^\star,t_2^\star)$、$P_{\text{succ}}$、$\mathbb{E}[\text{draws}]$、$\mathbb{E}[T_{\text{res}}]$、$J_2(t_1^\star,\Delta^\star)$;
  \item 敏感性:对 $(\lambda,\kappa,c_1,c_r,\alpha)$ 及命中阈值 $\tau$ 的稳健性比较;
  \item 可视化:$P_{\text{hit}}(t)$ 曲线(含推荐竖线)、$J_2(t_1,\Delta)$ 热力图等。
\end{itemize}

\section{实现与数值细节}
\subsection{设计矩阵与数值稳定}
\begin{itemize}
  \item 多项式次数 $d$ 一般取 $2$--$3$ 即可;如样本量充足可用样条替代;
  \item 预测时对 $\widehat{\mu}(x)$ 做区间裁剪(如 $[10^{-4},\,0.499]$);
  \item \textbf{GA-Local} 的 $\widehat{\sigma}(t)$ 用分箱(步长 \SI{0.5}{week})+ 邻近插值;若箱内样本极少导致方差不可估,则回退全局 $\hat\sigma$ 并裁剪到 $[0.005,0.06]$。
\end{itemize}

\subsection{区组与代表情景}
组别 $g$ 的 $P_{\text{hit}}(t\mid s_g)$ 用组内 BMI 中位数、总体(或组内)年龄/身高/体重/ QC 的中位数作为代表情景,以减少维度、提升可解释性。

\subsection{Bootstrap 与区间估计}
每次重采样在组内进行(对个体索引有放回抽样),\emph{重建} $\widehat{\mu},\widehat{\sigma}$ 并重算时点,自然反映了建模不确定性与组间异质性。CI 可用百分位数或 BCa。

\subsection{两次策略的相关性处理}
$\alpha=1$ 表示两次命中“近似独立”;$\alpha=0$ 则两次强相关(第二次命中概率退化为 $P_1$)。在不知道真实依赖结构时,建议报告若干 $\alpha$ 的敏感性结果以形成“策略区间”。

\section{小结}
\begin{itemize}
  \item Q3 用“\textbf{均值回归 + 误差结构}”的两层框架,将“胎儿成分阈值达标”问题转化为可随孕周连续评估的概率曲线 $P_{\text{hit}}(t\mid x)$;在 BMI 分组下定义“\textbf{最早达标}”与“\textbf{风险--命中权衡}”两类时点,并用 Bootstrap 与 $\sigma$ 扰动量化不确定性与误差影响。
  \item Q4 在此基础上构造“\textbf{一次/两次检测策略优化}”,将孕周风险、未达标惩罚、抽血成本与延迟成本统一进目标函数,通过网格搜索得到 $(t_1^\star,\Delta^\star)$ 并输出成功率、期望抽血次数与期望结果孕周等指标。
\end{itemize}

\end{document}
