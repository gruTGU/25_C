% !TEX program = xelatex
\documentclass[11pt,a4paper]{article}

% ---------- 基础版式 ----------
\usepackage[margin=1in]{geometry}
\usepackage{setspace}
\setstretch{1.15}
\usepackage{parskip}
\setlength{\parskip}{0.6em}
\setlength{\parindent}{0pt}

% ---------- XeLaTeX 字体与中文支持(macOS 适配) ----------
\usepackage{fontspec}
\usepackage{xeCJK}
\usepackage{ifthen}
\usepackage{xcolor}

% 英文字体(优先 Apple 系,缺省回退到 Times New Roman)
\IfFontExistsTF{Times New Roman}{
  \setmainfont{Times New Roman}
}{
  \setmainfont{Times}
}
\IfFontExistsTF{Helvetica Neue}{
  \setsansfont{Helvetica Neue}
}{
  \setsansfont{Arial}
}
\IfFontExistsTF{Menlo}{
  \setmonofont{Menlo}
}{
  \setmonofont{Courier New}
}

% 中文字体(按 macOS 常见中文字体依次回退)
\IfFontExistsTF{sans-serif}{
  \setCJKmainfont{sans-serif}
}{
  \IfFontExistsTF{STHeiti}{
    \setCJKmainfont{STHeiti}
  }{
    \IfFontExistsTF{WenQuanYi Zen Hei}{
      \setCJKmainfont{WenQuanYi Zen Hei}
    }{
      \setCJKmainfont{Arial Unicode MS}
    }
  }
}
\IfFontExistsTF{Heiti SC}{
  \setCJKsansfont{Heiti SC}
}{
  \setCJKsansfont{PingFang SC}
}
\IfFontExistsTF{Kaiti SC}{
  \setCJKmonofont{Kaiti SC}
}{
  \setCJKmonofont{Songti SC}
}

% 中文断行 & 数学内中文
\xeCJKsetup{CJKglue=\hskip 0.15em plus 0.05em minus 0.05em, CJKmath=true}

\documentclass[11pt,a4paper]{article}
\usepackage{amsmath, amssymb, amsfonts}
\usepackage{graphicx}
\usepackage{booktabs}
\usepackage{hyperref}
\usepackage[margin=1in]{geometry}

\title{NIPT 第四问:基于可解释机器学习的个体化检测与复检策略}
\author{项目:NIPT 的时点选择与胎儿的异常判定}
\date{\today}

\begin{document}
\maketitle

\section{目标}
在男胎样本上,利用可解释的概率模型给出首检孕周 $t_1$ 与(可选)复检孕周 $t_2$ 的最优策略,使
\[
J = \underbrace{\mathrm{risk}(T_{\mathrm{res}})}_{\text{孕周风险}}
+ \lambda\,(1-P_{\mathrm{succ}})
+ c_1 + \mathbf{1}\{\text{复检}\}\,c_r
+ \kappa\,\mathbb{E}[T_{\mathrm{res}}-t_{\min}]
\]
最小。这里 $\mathrm{risk}(t) \in \{1,2,3\}$ 分别对应 $t\!\le\!12$, $13\!\le\!t\!\le\!27$, $t\!\ge\!28$ 周。

\section{命中率模型 $P_{\mathrm{hit}}(t\mid x)$}
用可解释模型(GAM/EBM/LightGBM(单调约束))预测在孕周 $t$ 达到阈值(如 $Y\!\ge\!4\%$)的概率。
设特征 $x=(\text{GA},\text{BMI},\text{Age},\text{Height},\text{Weight})$,标签
$y=\mathbf{1}\{Y\ge 0.04\}$。

\paragraph{单调约束}
为符合生理先验,要求 $P_{\mathrm{hit}}$ 对 GA 单调不减:
\[
\frac{\partial}{\partial\,\mathrm{GA}} P_{\mathrm{hit}}(t\mid x) \ge 0.
\]
实际实现中在 LightGBM 里对 GA 特征施加正单调约束;GAM/EBM 则靠样条/分段函数与等概率校准实现近似单调。

\paragraph{概率校准}
先得原始分数 $s(x)$,再用单调的校准函数 $g(\cdot)$(如保序回归)得到
\[
\hat P_{\mathrm{hit}}(x) = g\big(s(x)\big),
\]
以提升概率可解释性。

\section{策略优化}
\subsection{单次检测}
在网格 $t\!\in[t_{\min}, t_{\max}]$ 上求
\[
J_1(t)= \mathrm{risk}(t) + \lambda\,(1-\hat P_{\mathrm{hit}}(t\mid \tilde x)) + c_1 + \kappa\,(t-t_{\min}),
\]
取最小点 $t_1^\star$。$\tilde x$ 为分组情景(如 BMI 组中位数)。

\subsection{一次复检}
设 $t_2=t_1+\Delta$,$\Delta\in\mathcal{D}$(如 $\{1,1.5,2,3\}$ 周)。一次成功概率 $P_1=\hat P(t_1)$;
二次命中概率 $P_2'=\alpha \hat P(t_2)+(1-\alpha)\hat P(t_1)$($\alpha\!\in[0,1]$ 控制独立性)。
成功概率 $P_{\mathrm{succ}}=P_1+(1-P_1)P_2'$,
期望结果孕周 $\mathbb{E}[T_{\mathrm{res}}]=P_1 t_1 + (1-P_1)t_2$。
目标
\[
J_2(t_1,\Delta)= c_1+(1-P_1)c_r + \mathbb{E}[\mathrm{risk}(T_{\mathrm{res}})] + \lambda\,(1-P_{\mathrm{succ}})
+ \kappa\,(\mathbb{E}[T_{\mathrm{res}}]-t_{\min}).
\]
在网格 $(t_1,\Delta)$ 上暴力搜索最优值。

\section{解释性}
输出特征贡献曲线(GAM/EBM 形状函数)、LightGBM 的单调 PDP、或 SHAP(如可用)。
并提供策略表(各 BMI 组的 $t_1^\*,t_2^\*$、成功率与置信带)。

\section{实现}
附带的 \texttt{nipt\_q4\_ml\_policy.py} 脚本实现上述流程,默认:
\begin{itemize}
\item 模型优先 LightGBM(单调) $\rightarrow$ EBM $\rightarrow$ GAM(样条GLM);
\item 校准优先 Isotonic $\rightarrow$ Platt $\rightarrow$ 无;
\item 输出策略表与可视化(热力图、曲线)。
\end{itemize}

\end{document}
